\subsection{How do I create a UDP
server?}\label{how-do-i-create-a-udp-server}

There are a variety of function calls available to send UDP sockets. We
will use the newer getaddrinfo to help set up a socket structure.

Remember that UDP is a simple packet-based (`data-gram') protocol ;
there is no connection to set up between the two hosts.

First, initialize the hints addrinfo struct to request an IPv6, passive
datagram socket.

\begin{Shaded}
\begin{Highlighting}[]
\NormalTok{memset(&hints, }\DecValTok{0}\NormalTok{, }\KeywordTok{sizeof}\NormalTok{(hints));}
\NormalTok{hints.ai_family = AF_INET6; }\CommentTok{// INET for IPv4}
\NormalTok{hints.ai_socktype =  SOCK_DGRAM;}
\NormalTok{hints.ai_flags =  AI_PASSIVE;}
\end{Highlighting}
\end{Shaded}

Next, use getaddrinfo to specify the port number (we don't need to
specify a host as we are creating a server socket, not sending a packet
to a remote host).

\begin{Shaded}
\begin{Highlighting}[]
\NormalTok{getaddrinfo(NULL, }\StringTok{"300"}\NormalTok{, &hints, &res);}

\NormalTok{sockfd = socket(res->ai_family, res->ai_socktype, res->ai_protocol);}
\NormalTok{bind(sockfd, res->ai_addr, res->ai_addrlen);}
\end{Highlighting}
\end{Shaded}

The port number is \textless{}1024, so the program will need
\texttt{root} privileges. We could have also specified a service name
instead of a numeric port value.

So far the calls have been similar to a TCP server. For a stream-based
service we would call \texttt{listen} and accept. For our UDP-serve we
can just start waiting for the arrival of a packet on the socket-

\begin{Shaded}
\begin{Highlighting}[]
\KeywordTok{struct} \NormalTok{sockaddr_storage addr;}
\DataTypeTok{int} \NormalTok{addrlen = }\KeywordTok{sizeof}\NormalTok{(addr);}

\CommentTok{// ssize_t recvfrom(int socket, void* buffer, size_t buflen, int flags, struct sockaddr *addr, socklen_t * address_len);}

\NormalTok{byte_count = recvfrom(sockfd, buf, }\KeywordTok{sizeof}\NormalTok{(buf), }\DecValTok{0}\NormalTok{, &addr, &addrlen);}
\end{Highlighting}
\end{Shaded}

The addr struct will hold sender (source) information about the arriving
packet.\\Note the \texttt{sockaddr\_storage} type is a sufficiently
large enough to hold all possible types of socket addresses (e.g.~IPv4,
IPv6 and other socket types).
