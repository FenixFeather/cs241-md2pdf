\chapter{Files}een
\href{http://angrave.github.io/sysassets/web/chapter1.html}{open} and
fopen (todo link here) so let's look at some more advanced concepts.

\section{How do I tell how large a file
is?}\label{how-do-i-tell-how-large-a-file-is}

For files less than the size of a long use fseek and ftell is a simple
way to accomplish this:

Move to the end of the file and find out the current position.

\begin{Shaded}
\begin{Highlighting}[]
\NormalTok{fseek(f, }\DecValTok{0}\NormalTok{, SEEK_END);}
\DataTypeTok{long} \NormalTok{pos = ftell(f);}
\end{Highlighting}
\end{Shaded}

This tells us the current position in the file in bytes - i.e.~the
length of the file!

\texttt{fseek} can also be used to set the absolute position.

\begin{Shaded}
\begin{Highlighting}[]
\NormalTok{fseek(f, }\DecValTok{0}\NormalTok{, SEEK_SET); }\CommentTok{// Move to the start of the file }
\NormalTok{fseek(f, posn, SEEK_SET);  }\CommentTok{// Move to 'posn' in the file.}
\end{Highlighting}
\end{Shaded}

All future reads and writes in the parent or child processes will be
honor this position.\\Note writing or reading from the file will change
the current position.

See the man pages for fseek and ftell for more information.

\subsection{\texorpdfstring{What happens if a child process closes a
filestream using \texttt{fclose} or
\texttt{close}?}{What happens if a child process closes a filestream using fclose or close?}}\label{what-happens-if-a-child-process-closes-a-filestream-using-fclose-or-close}

Unlike position, closing a file stream is unique to each process. Other
processes can continue to use their own file-handle.
