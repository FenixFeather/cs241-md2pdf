\subsection{The Hitchhiker's Guide to Debugging C
Programs}\label{the-hitchhikers-guide-to-debugging-c-programs}

Feel free to add anything that you found helpful in debugging C programs
including but not limited to, debugger usage, recognizing common error
types, gotchas, and effective googling tips.

\begin{enumerate}
\item
  Make your code modular using helper functions. If there is a repeated
  task (getting the pointers to contiguous blocks in MP2 for example),
  make them helper functions. And make sure each function does one thing
  very well, so that you don't have to debug twice.
\item
  Use assertions to make sure your code works up to a certain point --
  and importantly, to make sure you don't break it later. For example,
  if your data structure is a doubly linked list, you can do something
  like, assert(node-\textgreater{}size ==
  node-\textgreater{}next-\textgreater{}prev-\textgreater{}size) to
  assert that the next node has a pointer to the current node. You can
  also check the pointer is pointing to an expected range of memory
  address, not null, -\textgreater{}size is reasonable etc.\\The NDEBUG
  macro will disable all assertions, so don't forget to set that once
  you finish debugging.
  \url{http://www.cplusplus.com/reference/cassert/assert/}
\end{enumerate}

\subsubsection{GDB}\label{gdb}

Introduction: \url{http://www.cs.cmu.edu/~gilpin/tutorial/}

\paragraph{Setting breakpoints
programmatically}\label{setting-breakpoints-programmatically}

A very useful trick when debugging complex C programs with GDB is
setting breakpoints in the source code.

\begin{Shaded}
\begin{Highlighting}[]
\DataTypeTok{int} \NormalTok{main() \{}
    \DataTypeTok{int} \NormalTok{val = }\DecValTok{1}\NormalTok{;}
    \NormalTok{val = }\DecValTok{42}\NormalTok{;}
    \NormalTok{asm(}\StringTok{"int $3"}\NormalTok{); }\CommentTok{// set a breakpoint here}
    \NormalTok{val = }\DecValTok{7}\NormalTok{;}
\NormalTok{\}}
\end{Highlighting}
\end{Shaded}

\begin{verbatim}
$ gcc main.c -g -o main && ./main
(gdb) r
[...]
Program received signal SIGTRAP, Trace/breakpoint trap.
main () at main.c:6
6       val = 7;
(gdb) p val
$1 = 42
\end{verbatim}

\paragraph{Checking memory content}\label{checking-memory-content}

\url{http://www.delorie.com/gnu/docs/gdb/gdb_56.html}

For example,

\begin{Shaded}
\begin{Highlighting}[]
\DataTypeTok{int} \NormalTok{main() \{}
    \DataTypeTok{char} \NormalTok{bad_string[}\DecValTok{3}\NormalTok{] = \{'C', 'a', 't'\};}
    \NormalTok{printf(}\StringTok{"%s"}\NormalTok{, bad_string);}
\NormalTok{\}}
\end{Highlighting}
\end{Shaded}

\begin{verbatim}
$ gcc main.c -g -o main && ./main
$ Cat ZVQ� $
\end{verbatim}

\begin{verbatim}
(gdb) l
1   #include <stdio.h>
2   int main() {
3       char bad_string[3] = {'C', 'a', 't'};
4       printf("%s", bad_string);
5   }
(gdb) b 4
Breakpoint 1 at 0x100000f57: file main.c, line 4.
(gdb) r
[...]
Breakpoint 1, main () at main.c:4
4       printf("%s", bad_string);
(gdb) x/16xb bad_string
0x7fff5fbff9cd: 0x63    0x61    0x74    0xe0    0xf9    0xbf    0x5f    0xff
0x7fff5fbff9d5: 0x7f    0x00    0x00    0xfd    0xb5    0x23    0x89    0xff
\end{verbatim}

Here, by using the \texttt{x} command with parameters \texttt{16xb}, we
can see that starting at memory address \texttt{0x7fff5fbff9c} (value of
\texttt{bad\_string}), printf would actually see the following sequence
of bytes as a string because we provided a malformed string without a
null terminator.

\texttt{0x43\ 0x61\ 0x74\ 0xe0\ 0xf9\ 0xbf\ 0x5f\ 0xff\ 0x7f\ 0x00}
