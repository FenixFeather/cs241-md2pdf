\subsection{Does a directory have an inode
too?}\label{does-a-directory-have-an-inode-too}

Yes! Though a better way to think about this, is that a directory (like
a file) \emph{is} an inode (with some data - the directory name and
inode contents). It just happens to be a special kind of inode.

\subsection{How can I have the same file appear in two different places
in my file
system?}\label{how-can-i-have-the-same-file-appear-in-two-different-places-in-my-file-system}

First remember that a file name != the file. Think of the inode as `the
file' and a directory as just a list of names with each name mapped to
an inode number. Some of those inodes may be regular file inodes, others
may be directory inodes.

If we already have a file on a file system we can create another link to
the same inode using the `ln' command

\begin{verbatim}
$ ln file1.txt blip.txt
\end{verbatim}

However blip.txt \emph{is} the same file; if I edit blip I'm editing the
same file as `file1.txt!'\\We can prove this by showing that both file
names refer to the same inode:

\begin{verbatim}
$ ls -i file1.txt blip.txt
134235 file1.txt
134235 blip.txt
\end{verbatim}

These kinds of links (aka directory entries) are called `hard links'

The equivalent C call is \texttt{link}

\begin{Shaded}
\begin{Highlighting}[]
\NormalTok{link(}\DataTypeTok{const} \DataTypeTok{char} \NormalTok{*path1, }\DataTypeTok{const} \DataTypeTok{char} \NormalTok{*path2);}

\NormalTok{link(}\StringTok{"file1.txt"}\NormalTok{, }\StringTok{"blip.txt"}\NormalTok{);}
\end{Highlighting}
\end{Shaded}

For simplicity the above examples made hard links inside the same
directory however hard links can be created anywhere inside the same
filesystem.

\subsection{\texorpdfstring{What happens when I \texttt{rm} (remove) a
file?}{What happens when I rm (remove) a file?}}\label{what-happens-when-i-rm-remove-a-file}

When you remove a file (using \texttt{rm} or \texttt{unlink}) you are
removing an inode reference from a directory.\\However the inode may
still be referenced from other directories. In order to determine if the
contents of the file are still required, each inode keeps a reference
count that is updated whenever a new link is created or destroyed.

\subsection{Case study: Back up software that minimizes file
duplication}\label{case-study-back-up-software-that-minimizes-file-duplication}

An example use of hard-links is to efficiently create multiple archives
of a file system at different points in time. Once the archive area has
a copy of a particular file, then future archives can re-use these
archive files rather than creating a duplicate file. Apple's ``Time
Machine'' software does this.

\subsection{Can I create hard links to directories as well as regular
files?}\label{can-i-create-hard-links-to-directories-as-well-as-regular-files}

No. Well yes. Not really\ldots{} Actually you didn't really want to do
this, did you?\\The POSIX standard says no you may not! The \texttt{ln}
command will only allow root to do this and only if you provide the
\texttt{-d} option. However even root may not be able to perform this
because most filesystems prevent it!

Why?\\The integrity of the file system assumes the directory structure
(excluding softlinks which we will talk about later) is a non-cyclic
tree that is reachable from the root directory. It becomes expensive to
enforce or verify this constraint if directory linking is allowed.
Breaking these assumptions can cause file integrity tools to not be able
to repair the file system. Recursive searches potentially never
terminate and directories can have more than one parent but ``..'' can
only refer to a single parent. All in all, a bad idea.

\subsection{How do I change the permissions on a
file?}\label{how-do-i-change-the-permissions-on-a-file}

Use \texttt{chmod} (short for ``change the file mode bits'')

There is a system call,
\texttt{int\ chmod(const\ char\ *path,\ mode\_t\ mode);} but we will
concentrate on the shell command. There's two common ways to use
\texttt{chmod} ; with an octal value or with a symbolic string:

\begin{verbatim}
$ chmod 644 file1
$ chmod 755 file2
$ chmod 700 file3
$ chmod ugo-w file4
$ chmod o-rx file4
\end{verbatim}

The base-8 (`octal') digits describe the permissions for each role: The
user who owns the file, the group and everyone else. The octal number is
the sum of three values given to the three types of permission: read(4),
write(2), execute(1)

Example: chmod 755 myfile

\begin{itemize}
\itemsep1pt\parskip0pt\parsep0pt
\item
  r + w + x = digit
\item
  user has 4+2+1, full permission
\item
  group has 4+0+1, read and execute permission
\item
  all users have 4+0+1, read and execute permission
\end{itemize}

\subsection{How do I read the permission string from
ls?}\label{how-do-i-read-the-permission-string-from-ls}

Use `ls -l'.\\Note that the permissions will output in the format
`drwxrwxrwx'. The first character indicates the type of file
type.\\Possible values for the first character:

\begin{itemize}
\itemsep1pt\parskip0pt\parsep0pt
\item
  (-) regular file
\item
  (d) directory
\item
  (c) character device file\textbackslash{}
\item
  (l) symbolic link
\item
  (p) pipe
\item
  (b) block device
\item
  (s) socket
\end{itemize}

\subsection{What is sudo?}\label{what-is-sudo}

Use \texttt{sudo} to become the admin on the machine.\\e.g.~Normally
(unless explicitly specified in the `/etc/fstab' file, you need root
access to mount a filesystem). \texttt{sudo} can be used to temporarily
run a command as root (provided the user has sudo privileges)

\begin{verbatim}
$ sudo mount /dev/sda2 /stuff/mydisk
$ sudo adduser fred
\end{verbatim}

\subsection{How do I change ownership of a
file?}\label{how-do-i-change-ownership-of-a-file}

Use \texttt{chown\ username\ filename}

\subsection{How do I set permissions from
code?}\label{how-do-i-set-permissions-from-code}

\texttt{chmod(const\ char\ *path,\ mode\_t\ mode);}

\subsection{\texorpdfstring{Why are some files `setuid' what does this
mean?
?}{Why are some files setuid what does this mean? ?}}\label{why-are-some-files-setuid-what-does-this-mean}

The set-user-ID-on-execution bit changes the user associated with the
process when the file is run. This is typically used for commands that
need to run as root but are executed by non-root users. An example of
this is \texttt{sudo}

The set-group-ID-on-execution changes the group under which the process
is run.

\subsection{Why are they useful?}\label{why-are-they-useful}

The most common usecase is so that the user can have root(admin) access
for the duration of the program.

\subsection{What permissions does sudo run as
?}\label{what-permissions-does-sudo-run-as}

\begin{verbatim}
$ ls -l /usr/bin/sudo
-r-s--x--x  1 root  wheel  327920 Oct 24 09:04 /usr/bin/sudo
\end{verbatim}

The `s' bit means execute and set-uid; the effective userid of the
process will be different from the parent process. In this example it
will be root

\subsection{What's the difference between getuid() and
geteuid()?}\label{whats-the-difference-between-getuid-and-geteuid}

\begin{itemize}
\itemsep1pt\parskip0pt\parsep0pt
\item
  \texttt{getuid} returns the real user id (zero if logged in as root)
\item
  \texttt{geteuid} returns the effective userid (zero if acting as root,
  e.g.~due to the setuid flag set on a program)
\end{itemize}

\subsection{How do I ensure only privileged users can run my
code?}\label{how-do-i-ensure-only-privileged-users-can-run-my-code}

\begin{itemize}
\itemsep1pt\parskip0pt\parsep0pt
\item
  Check the effective permissions of the user by calling
  \texttt{geteuid()}. A return value of zero means the program is
  running effectively as root.
\end{itemize}
