\subsection{\texorpdfstring{How do I use \texttt{getaddrinfo} to convert
the hostname into an IP
address?}{How do I use getaddrinfo to convert the hostname into an IP address?}}\label{how-do-i-use-getaddrinfo-to-convert-the-hostname-into-an-ip-address}

The function \texttt{getaddrinfo} can convert a human readable domain
name (e.g. \texttt{www.illinois.edu}) into an IPv4 and IPv6 address. In
fact it will return a linked-list of addrinfo structs:

\begin{Shaded}
\begin{Highlighting}[]
\KeywordTok{struct} \NormalTok{addrinfo \{}
    \DataTypeTok{int}              \NormalTok{ai_flags;}
    \DataTypeTok{int}              \NormalTok{ai_family;}
    \DataTypeTok{int}              \NormalTok{ai_socktype;}
    \DataTypeTok{int}              \NormalTok{ai_protocol;}
    \NormalTok{socklen_t        ai_addrlen;}
    \KeywordTok{struct} \NormalTok{sockaddr *ai_addr;}
    \DataTypeTok{char}            \NormalTok{*ai_canonname;}
    \KeywordTok{struct} \NormalTok{addrinfo *ai_next;}
\NormalTok{\};}
\end{Highlighting}
\end{Shaded}

It's very easy to use. For example, suppose you wanted to find out the
numeric IPv4 address of a webserver at www.bbc.com. We do this in two
stages. First use getaddrinfo to build a linked-list of possible
connections. Secondly use \texttt{getnameinfo} to convert the binary
address into a readable form.

\begin{Shaded}
\begin{Highlighting}[]
\OtherTok{#include <stdio.h>}
\OtherTok{#include <stdlib.h>}
\OtherTok{#include <sys/types.h>}
\OtherTok{#include <sys/socket.h>}
\OtherTok{#include <netdb.h>}

\KeywordTok{struct} \NormalTok{addrinfo hints, *infoptr; }\CommentTok{// So no need to use memset global variables}

\DataTypeTok{int} \NormalTok{main() \{}
  \NormalTok{hints.ai_family = AF_INET; }\CommentTok{// AF_INET means IPv4 only addresses}

  \DataTypeTok{int} \NormalTok{result = getaddrinfo(}\StringTok{"www.bbc.com"}\NormalTok{, NULL, &hints, &infoptr);}
  \KeywordTok{if} \NormalTok{(result) \{}
    \NormalTok{fprintf(stderr, }\StringTok{"getaddrinfo: %s}\CharTok{\textbackslash{}n}\StringTok{"}\NormalTok{, gai_strerror(result));}
    \NormalTok{exit(}\DecValTok{1}\NormalTok{);}
  \NormalTok{\}}

  \KeywordTok{struct} \NormalTok{addrinfo *p;}
  \DataTypeTok{char} \NormalTok{host[}\DecValTok{256}\NormalTok{],service[}\DecValTok{256}\NormalTok{];}

  \KeywordTok{for}\NormalTok{(p = infoptr; p != NULL; p = p->ai_next) \{}

    \NormalTok{getnameinfo(p->ai_addr, p->ai_addrlen, host, }\KeywordTok{sizeof}\NormalTok{(host), service, }\KeywordTok{sizeof}\NormalTok{(service), NI_NUMERICHOST);}
    \NormalTok{puts(host);}
  \NormalTok{\}}

  \NormalTok{freeaddrinfo(infoptr);}
  \KeywordTok{return} \DecValTok{0}\NormalTok{;}
\NormalTok{\}}
\end{Highlighting}
\end{Shaded}

Typical output:

\begin{verbatim}
212.58.244.70
212.58.244.71
\end{verbatim}

\subsection{How is www.cs.illinois.edu converted into an IP
address?}\label{how-is-www.cs.illinois.edu-converted-into-an-ip-address}

Magic! No seriously, a system called ``DNS'' (Domain Name Service) is
used. If a machine does not hold the answer locally then it sends a UDP
packet to a local DNS server. This server in turn may query other
upstream DNS servers.

\subsection{Is DNS secure?}\label{is-dns-secure}

DNS by itself is fast but not secure. DNS requests are not encrypted and
susceptible to `man-in-the-middle' attacks. For example, a coffee shop
internet connection could easily subvert your DNS requests and send back
different IP addresses for a particular domain

\subsection{How do I connect to a TCP server (e.g.~web
server?)}\label{how-do-i-connect-to-a-tcp-server-e.g.web-server}

TODO\\There are three basic system calls you need to connect to a remote
machine:

\begin{verbatim}
getaddrinfo -- Determine the remote addresses of a remote host
socket  -- Create a socket
connect  -- Connect to the remote host using the socket and address information
\end{verbatim}

The \texttt{getaddrinfo} call if successful, creates a linked-list of
\texttt{addrinfo} structs and sets the given pointer to point to the
first one.

The socket call creates an outgoing socket and returns a descriptor
(sometimes called a `file descriptor') that can be used with
\texttt{read} and \texttt{write} etc.In this sense it is the network
analog of \texttt{open} that opens a file stream - except that we
haven't connected the socket to anything yet!

Finally the connect call attempts the connection to the remote machine.
We pass the original socket descriptor and also the socket address
information which is stored inside the addrinfo structure. There are
different kinds of socket address structures (e.g.~IPv4 vs IPv6) which
can require more memory. So in addition to passing the pointer, the size
of the structure is also passed:

\begin{Shaded}
\begin{Highlighting}[]
\CommentTok{// Pull out the socket address info from the addrinfo struct:}
\NormalTok{connect(sockfd, p->ai_addr, p->ai_addrlen)}
\end{Highlighting}
\end{Shaded}

\subsection{How do I free the memory allocated for the linked-list of
addrinfo
structs?}\label{how-do-i-free-the-memory-allocated-for-the-linked-list-of-addrinfo-structs}

As part of the clean up code call \texttt{freeaddrinfo} on the top-most
\texttt{addrinfo} struct:

\begin{Shaded}
\begin{Highlighting}[]
\DataTypeTok{void} \NormalTok{freeaddrinfo(}\KeywordTok{struct} \NormalTok{addrinfo *ai);}
\end{Highlighting}
\end{Shaded}

\subsection{\texorpdfstring{If getaddrinfo fails can I use
\texttt{strerror} to print out the
error?}{If getaddrinfo fails can I use strerror to print out the error?}}\label{if-getaddrinfo-fails-can-i-use-strerror-to-print-out-the-error}

No. Error handling with \texttt{getaddrinfo} is a little different:

\begin{itemize}
\itemsep1pt\parskip0pt\parsep0pt
\item
  The return value \emph{is} the error code (i.e.~don't use
  \texttt{errno})
\item
  Use \texttt{gai\_strerror} to get the equivalent short English error
  text:
\end{itemize}

\begin{Shaded}
\begin{Highlighting}[]
\DataTypeTok{int} \NormalTok{result = getaddrinfo(...);}
\KeywordTok{if}\NormalTok{(result) \{ }
   \DataTypeTok{char} \NormalTok{*mesg = gai_strerror(result); }
   \NormalTok{...}
\NormalTok{\}}
\end{Highlighting}
\end{Shaded}

\subsection{Can I request only IPv4 or IPv6 connection? TCP
only?}\label{can-i-request-only-ipv4-or-ipv6-connection-tcp-only}

Yes! Use the addrinfo structure that is passed into \texttt{getaddrinfo}
to define the kind of connection you'd like.

For example, to specify stream-based protocols over IPv6:

\begin{Shaded}
\begin{Highlighting}[]
\KeywordTok{struct} \NormalTok{addrinfo hints;}
\NormalTok{memset(hints, }\DecValTok{0}\NormalTok{, }\KeywordTok{sizeof}\NormalTok{(hints));}

\NormalTok{hints.ai_family = AF_INET6; }\CommentTok{// Only want IPv6 (use AF_INET for IPv4)}
\NormalTok{hints.ai_socktype = SOCK_STREAM; }\CommentTok{// Only want stream-based connection}
\end{Highlighting}
\end{Shaded}

\subsection{\texorpdfstring{What about code examples that use
\texttt{gethostbyname}?}{What about code examples that use gethostbyname?}}\label{what-about-code-examples-that-use-gethostbyname}

The old function \texttt{gethostbyname} is deprecated; it's the old way
convert a host name into an IP address. The port address still needs to
be manually set using htons function. It's much easier to write code to
support IPv4 AND IPv6 using the newer \texttt{getaddrinfo}

\subsection{Is it that easy!?}\label{is-it-that-easy}

Yes and no. It's easy to create a simple TCP client - however network
communications offers many different levels of abstraction and several
attributes and options that can be set at each level of abstraction (for
example we haven't talked about \texttt{setsockopt} which can manipulate
options for the socket).\\For more information
see\\{[}{[}\url{http://www.beej.us/guide/bgnet/output/html/multipage/getaddrinfoman.html}{]}{]}
