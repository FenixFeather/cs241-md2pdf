\subsection{What is the Critical Section
Problem?}\label{what-is-the-critical-section-problem}

As already discussed in {[}{[}Synchronization, Part 3: Working with
Mutexes And Semaphores{]}{]}, there are critical parts of our code that
can only be executed by one thread at a time. We describe this
requirement as `mutual exclusion'; only one thread (or process) may have
access to the shared resource.

In multi-threaded programs we can wrap a critical section with mutex
lock and unlock calls:

\begin{Shaded}
\begin{Highlighting}[]
\NormalTok{pthread_mutex_lock() - one thread allowed at a time! (others will have to wait here)}
\NormalTok{... Do Critical Section stuff here!}
\NormalTok{pthread_mutex_unlock() - let other waiting threads }\KeywordTok{continue}
\end{Highlighting}
\end{Shaded}

How would we implement these lock and unlock calls? Can we create an
algorithm that assures mutual exclusion? An incorrect implementation is
shown below,

\begin{Shaded}
\begin{Highlighting}[]
\NormalTok{pthread_mutex_lock(p_mutex_t *m)     \{ }\KeywordTok{while}\NormalTok{(m->lock) \{\}; m->lock = }\DecValTok{1}\NormalTok{;\}}
\NormalTok{pthread_mutex_unlock(p_mutex_t *m)   \{ m->lock = }\DecValTok{0}\NormalTok{; \}}
\end{Highlighting}
\end{Shaded}

At first glance, the code appears to work; if one thread attempts to
locks the mutex, a later thread must wait until the lock is cleared.
However this implementation \emph{does not satisfy Mutual Exclusion}.
Let's take a close look at this `implementation' from the point of view
of two threads running around the same time. In the table below times
runs from top to bottom-

Time \textbar{} Thread 1 \textbar{} Thread
2\\-----\textbar{}----------\textbar{}---------\\1 \textbar{}
\texttt{while(lock)\ \{\}}\\2 \textbar{} \textbar{}
\texttt{while(lock)\ \{\}} \textbar{}\\3 \textbar{} lock = 1 \textbar{}
lock = 1 \textbar{}\\Ooops! There is a race condition. In the
unfortunate case both threads checked the lock and read a false value
and so were able to continue.

\subsection{Candidate solutions to the critical section
problem.}\label{candidate-solutions-to-the-critical-section-problem.}

To simplify the discussion we consider only two threads. Note these
arguments work for threads and processes and the classic CS literature
discusses these problem in terms of two processes that need exclusive
access (i.e.~mutual exclusion) to a critical section or shared resource.

Remember that the psuedo-code outlined below is part of a larger
program; the thread or process will typically need to enter the critical
section many times during the lifetime of the process. So imagine each
example as wrapped inside a loop where for a random amount of time the
thread or process is working on something else.

Is there anything wrong with candidate solution described below?

\begin{verbatim}
// Candidate #1
wait until your flag is lowered
raise my flag
// Do Critical Section stuff
lower my flag 
\end{verbatim}

Answer: Candidate solution \#1 also suffers a race condition i.e.~it
does not satisfy Mutual Exclusion because both threads/processes could
read each other's flag value (=lowered) and continue.

This suggests we should raise the flag \emph{before} checking the other
thread's flag - which is candidate solution \#2 below.

\begin{verbatim}
// Candidate #2
raise my flag
wait until your flag is lowered
// Do Critical Section stuff
lower my flag 
\end{verbatim}

Candidate \#2 satisfies mutual exclusion - it is impossible for two
threads to be inside the critical section at the same time. However this
code suffers from deadlock! Suppose two threads wish to enter the
critical section at the same time:

Time \textbar{} Thread 1 \textbar{} Thread
2\\-----\textbar{}----------\textbar{}---------\\1 \textbar{} raise
flag\\2 \textbar{} \textbar{} raise flag\\3 \textbar{} wait \ldots{}
\textbar{} wait \ldots{}\\Ooops both threads / processes are now waiting
for the other one to lower their flags. Neither one will enter the
critical section as both are now stuck forever!

This suggests we should use a turn-based variable to try to resolve who
should proceed.

\subsection{Turn-based solutions}\label{turn-based-solutions}

The following candidate solution \#3 uses a turn-based variable to
politely allow one thread and then the other to continue

\begin{verbatim}
// Candidate #3
wait until my turn is myid
// Do Critical Section stuff
turn = yourid
\end{verbatim}

Candidate \#3 satisfies mutual exclusion (each thread or process gets
exclusive access to the Critical Section), however both
threads/processes must take a strict turn-based approach to using the
critical section. For example, if thread 1 wishes to read a hashtable
every millisecond but another thread writes to a hashtable every second,
then the reading thread would have to wait another 999ms before being
able to read from the hashtable again. This `solution' is not effective,
because our threads should be able to make progress and enter the
critical section if no other thread is currently in the critical
section.

\subsection{Desired properties for solutions to the Critical Section
Problem?}\label{desired-properties-for-solutions-to-the-critical-section-problem}

There are three main desirable properties that we desire in a solution
the critical section problem

\begin{itemize}
\itemsep1pt\parskip0pt\parsep0pt
\item
  Mutual Exclusion - the thread/process gets exclusive access; others
  must wait until it exits the critical section.
\item
  Bounded Wait - if the thread/process has to wait, then it should only
  have to wait for a finite, amount of time (infinite waiting times are
  not allowed!). The exact definition of bounded wait is that there is
  an upper (non-infinite) bound on the number of times any other process
  can enter its critical section before the given process enters.
\item
  Progress - if no thread/process is inside the critical section, then
  the thread/process should be able to proceed (make progress) without
  having to wait.
\end{itemize}

With these ideas in mind let's examine another candidate solution that
uses a turn-based flag only if two threads both required access at the
same time.

\subsection{Turn and Flag solutions}\label{turn-and-flag-solutions}

Is the following a correct solution to CSP?

\begin{verbatim}
\\ Candidate #4
raise my flag
if your flag is raised, wait until my turn
// Do Critical Section stuff
turn = yourid
lower my flag
\end{verbatim}

One instructor and another CS faculty member initially thought so!
However analyzing these solutions are tricky. Even peer-reviewed papers
on this specific subject contain incorrect solutions! At first glance it
appears to satisfy Mutual Exclusion, Bounded Wait and Progress: The
turn-based flag is only used in the event of a tie (so Progress and
Bounded Wait is allowed) and mutual exclusion appears to be satisfied.
However\ldots{}. Perhaps you can find a counter-example?

Candidate \#4 fails because a thread does not wait until the other
thread lowers their flag. After some thought (or inspiration) the
following scenario can be created to demonstrate how Mutual Exclusion is
not satisfied.

Imagine the first thread runs this code twice (so the the turn flag now
points to the second thread). While the first thread is still inside the
Critical Section, the second thread arrives. The second thread can
immediately continue into the Critical Section!

\begin{longtable}[c]{@{}llll@{}}
\toprule
Time & Turn & Thread \#1 & Thread \#2\tabularnewline
\midrule
\endhead
1 & 2 & raise my flag\tabularnewline
2 & 2 & if your flag is raised, wait until my turn & raise my
flag\tabularnewline
3 & 2 & // Do Critical Section stuff & if your flag is raised, wait
until my turn(TRUE!)\tabularnewline
4 & 2 & // Do Critical Section stuff & // Do Critical Section stuff -
OOPS\tabularnewline
\bottomrule
\end{longtable}

\subsection{What is Peterson's
solution?}\label{what-is-petersons-solution}

Peterson published his novel and surprisingly simple solution in a 2
page paper in 1981. A version of his algorithm is shown below that uses
a shared variable `turn' -

\begin{verbatim}
\\ Candidate #5
raise my flag
turn = myid
wait until your flag is lowered or turn is yourid
// Do Critical Section stuff
lower my flag
\end{verbatim}

This solution satisfies Mutual Exclusion, Bounded Wait and Progress. If
thread \#2 has set turn to 2 and is currently inside the critical
section. Thread \#1 arrives, \emph{sets the turn back to 1} and now
waits until thread 2 lowers the flag.

Link to Peterson's original article pdf:\\{[}{[}G. L. Peterson: ``Myths
About the Mutual Exclusion Problem'', Information Processing Letters
12(3) 1981,
115--116\textbar{}\url{http://dl.acm.org/citation.cfm?id=945527}{]}{]}

\subsection{Was Peterson's solution the first
solution?}\label{was-petersons-solution-the-first-solution}

No. Todo; Discuss Dekkers Algorithm

\begin{verbatim}
raise my flag
while(your flag is raised) :
   if it's your turn to win :
     lower my flag
     wait while your turn
     raise my flag
// Do Critical Section stuff
set your turn to win
lower my flag
\end{verbatim}

\subsection{Can I implement Peterson's algorithm in C or
assembler?}\label{can-i-implement-petersons-algorithm-in-c-or-assembler}

Yes - and it is used today in production for specific mobile processors:
Peterson's algorithm is used to implement low-level Linux Kernel locks
for the Tegra mobile processor (a system-on-chip ARM process and GPU
core by
Nvidia)\\\url{https://android.googlesource.com/kernel/tegra.git/+/android-tegra-3.10/arch/arm/mach-tegra/sleep.S\#58}

However in general, CPUs and C compilers can re-order CPU instructions
or use CPU-core-specific local cache values that are stale if another
core updates the shared variables. Thus a simple pseudo-code to C
implementation is too naive for most platforms.

\subsection{How do we implement Critical Section Problem on
hardware?}\label{how-do-we-implement-critical-section-problem-on-hardware}

Good question. Next lecture\ldots{}
