\subsection{Warm up}\label{warm-up}

Name these properties!

\begin{itemize}
\itemsep1pt\parskip0pt\parsep0pt
\item
  ``Only one process(/thread) can be in the CS at a time''
\item
  ``If waiting, then another process can only enter the CS a finite
  number of times''
\item
  ``If no other process is in the CS then the process can immediately
  enter the CS''
\end{itemize}

See {[}{[}Synchronization, Part 4: The Critical Section Problem{]}{]}
for answers.

\subsection{\texorpdfstring{What is the `exchange instruction'
?}{What is the exchange instruction ?}}\label{what-is-the-exchange-instruction}

The exchange instruction (`XCHG') is an atomic CPU instruction that
exchanges the contents of a register with a memory location. This can be
used as a basis to implement a simple mutex lock.

\begin{Shaded}
\begin{Highlighting}[]
\CommentTok{// *Pseudo-C-code* for a simple busy-waiting mutex }
\CommentTok{// that uses an atomic exchange function}
\DataTypeTok{int} \NormalTok{lock = }\DecValTok{0}\NormalTok{; }\CommentTok{// initialization}

\CommentTok{// To enter the critical section you need to read a lock value of zero. }
\CommentTok{// 'xchg' function doesn't exist, but imagine this function is built on the atomic XCHG CPU function}
\CommentTok{// i.e. it writes '1' into the lock variable and returns the previous contents of the memory}
\KeywordTok{while} \NormalTok{(xchg( }\DecValTok{1}\NormalTok{, &lock)) \{}\CommentTok{/*spin spin spin*/}\NormalTok{\}}
\CommentTok{/* Do Critical Section stuff*/}
\NormalTok{lock = }\DecValTok{0}\NormalTok{;}
\end{Highlighting}
\end{Shaded}

\subsection{What are condition variables? How do you use them? What is
Spurious
Wakeup?}\label{what-are-condition-variables-how-do-you-use-them-what-is-spurious-wakeup}

\begin{itemize}
\item
  Condition variables allow a set of threads to sleep until tickled! You
  can tickle one thread or all threads that are sleeping. If you only
  wake one thread then the operating system will decide which thread to
  wake up. You don't wake threads directly instead you `signal' the
  condition variable, which then will wake up one (or all) threads that
  are sleeping inside the condition variable.
\item
  Condition variables are used with a mutex and with a loop (to check a
  condition).
\item
  Occasionally a waiting thread may appear to wake up for no reason
  (this is called a \emph{spurious wake})! This is not an issue because
  you always use \texttt{wait} inside a loop that tests a condition that
  must be true to continue.
\item
  Threads sleeping inside a condition variable are woken up calling
  \texttt{pthread\_cond\_broadcast} (wake up all) or
  \texttt{pthread\_cond\_signal} (wake up one). Note despite the
  function name, this has nothing to do with POSIX \texttt{signal}s!
\end{itemize}

\subsection{\texorpdfstring{What does \texttt{pthread\_cond\_wait}
do?}{What does pthread\_cond\_wait do?}}\label{what-does-pthreadux5fcondux5fwait-do}

Wait performs three actions:

\begin{itemize}
\itemsep1pt\parskip0pt\parsep0pt
\item
  unlock the mutex and atomically\ldots{}
\item
  waits (sleeps until \texttt{pthread\_cond\_signal} is called on the
  same condition variable)
\item
  Before returning, locks the mutex
\end{itemize}

\subsection{(Advanced topic) Why do Condition Variables also need a
mutex?}\label{advanced-topic-why-do-condition-variables-also-need-a-mutex}

Condition variables need a mutex for three reasons. The simplest to
understand is that it prevents an early wakeup message (\texttt{signal}
or \texttt{broadcast} functions) from being `lost.' Imagine the
following sequence of events (time runs down the page) where the
condition is satisfied \emph{just before }\texttt{pthread\_cond\_wait}
is called. In this example the wake-up signal is lost!

Thread 1 \textbar{} Thread
2\\-------------------------\textbar{}---------\\\texttt{while(\ answer\ \textless{}\ 42)\ \{}
\textbar{}\\ \textbar{} \texttt{answer++}\\ \textbar{}
\texttt{p\_cond\_signal(cv)}\\\texttt{p\_cond\_wait(cv,m)} \textbar{}

If both threads had locked a mutex, the signal can not be sent until
\emph{after} \texttt{pthread\_cond\_wait(cv,\ m)} is called (which then
internally unlocks the mutex)

A second common reason is that updating the program state
(\texttt{answer} variable) typically requires mutual exclusion - for
example multiple threads may be updating the value of \texttt{answer}.

A third and subtle reason is to satisfy real-time scheduling concerns
which we only outline here: In a time-critical application, the waiting
thread with the \emph{highest priority} should be allowed to continue
first. To satisfy this requirement the mutex must also be locked before
calling \texttt{pthread\_cond\_signal} or
\texttt{pthread\_cond\_broadcast} . For the curious, a longer and
historical discussion is
{[}{[}here\textbar{}\url{https://groups.google.com/forum/?hl=ky}\#!msg/comp.programming.threads/wEUgPq541v8/ZByyyS8acqMJ{]}{]}.

\subsection{Why do spurious wakes
exist?}\label{why-do-spurious-wakes-exist}

For performance. On multi-CPU systems it is possible that a
race-condition could cause a wake-up (signal) request to be unnoticed.
The kernel may not detect this lost wake-up call but can detect when it
might occur. To avoid the potential lost signal the thread is woken up
so that the program code can test the condition again.

\subsection{Example}\label{example}

Condition variables are \emph{always} used with a mutex lock.

Before calling \emph{wait}, the mutex lock must be locked and
\emph{wait} must be wrapped with a loop.

\begin{Shaded}
\begin{Highlighting}[]
\NormalTok{pthread_cond_t cv;}
\NormalTok{pthread_mutex_t m;}
\DataTypeTok{int} \NormalTok{count;}

\CommentTok{// Initialize}
\NormalTok{pthread_cond_init(&cv, NULL);}
\NormalTok{pthread_mutex_init(&m, NULL);}
\NormalTok{count = }\DecValTok{0}\NormalTok{;}

\NormalTok{pthread_mutex_lock(&m);}
\KeywordTok{while} \NormalTok{(count < }\DecValTok{10}\NormalTok{) \{}
   \NormalTok{pthread_cond_wait(&cv, &m); }\CommentTok{/*unlock m,wait, lock m*/}
\NormalTok{\}}
\NormalTok{pthread_mutex_unlock(&m);}

\CommentTok{//later clean up with pthread_cond_destroy(&cv); and mutex_destroy }


\CommentTok{// In another thread increment count:}
\KeywordTok{while} \NormalTok{(}\DecValTok{1}\NormalTok{) \{}
  \NormalTok{pthread_mutex_lock(&m);}
  \NormalTok{count++;}
  \NormalTok{pthread_cond_signal(&cv);}
  \CommentTok{/* Even though the other thread is woken up it will not return}
\CommentTok{  we unlock the mutex */}
  \NormalTok{pthread_mutex_unlock(&m);}
\NormalTok{\}}
\end{Highlighting}
\end{Shaded}

\subsection{Implementing counting
semaphores}\label{implementing-counting-semaphores}

\begin{itemize}
\item
  We can implement a counting semaphore using condition variables.
\item
  Each semaphore needs a count, a condition variable and a mutex

\begin{Shaded}
\begin{Highlighting}[]
\KeywordTok{typedef} \KeywordTok{struct} \NormalTok{sem_t \{}
  \DataTypeTok{int} \NormalTok{count; }
  \NormalTok{pthread_mutex_t m;}
  \NormalTok{pthread_condition_t cv;}
\NormalTok{\} sem_t;}
\end{Highlighting}
\end{Shaded}
\end{itemize}

Implement \texttt{sem\_init} to initialize the mutex and condition
variable

\begin{Shaded}
\begin{Highlighting}[]
\DataTypeTok{int} \NormalTok{sem_init(sem_t *s, }\DataTypeTok{int} \NormalTok{pshared, }\DataTypeTok{int} \NormalTok{value) \{}
  \KeywordTok{if} \NormalTok{(pshared) \{ errno = ENOSYS }\CommentTok{/* 'Not implemented'*/}\NormalTok{; }\KeywordTok{return} \NormalTok{-}\DecValTok{1}\NormalTok{;\}}

  \NormalTok{s->count = value;}
  \NormalTok{pthread_mutex_init(&s->m, NULL);}
  \NormalTok{pthread_condition_init(&s->cv, NULL);}
  \KeywordTok{return} \DecValTok{0}\NormalTok{;}
\NormalTok{\}}
\end{Highlighting}
\end{Shaded}

Our implementation of \texttt{sem\_post} needs to increment the
count.\\We will also wake up any threads sleeping inside the condition
variable.\\Notice we lock and unlock the mutex so only one thread can be
inside the critical section at a time.

\begin{Shaded}
\begin{Highlighting}[]
\NormalTok{sem_post(sem_t *s) \{}
  \NormalTok{pthread_mutex_lock(&s->m);}
  \NormalTok{s->count++;}
  \NormalTok{pthread_cond_signal(s->cv); }\CommentTok{/* See note */}
  \CommentTok{/* A woken thread must acquire the lock, so it will also have to wait until we call unlock*/}

  \NormalTok{pthread_mutex_unlock(&s->m);}
\NormalTok{\}}
\end{Highlighting}
\end{Shaded}

Our implementation of \texttt{sem\_wait} may need to sleep if the
semaphore's count is zero.\\Just like \texttt{sem\_post} we wrap the
critical section using the lock (so only one thread can be executing our
code at a time). Notice if the thread does need to wait then the mutex
will be unlocked, allowing another thread to enter \texttt{sem\_post}
and waken us from our sleep!

Notice that even if a thread is woken up, before it returns from
\texttt{pthread\_cond\_wait} it must re-acquire the lock, so it will
have to wait a little bit more (e.g.~until sem\_post finishes).

\begin{Shaded}
\begin{Highlighting}[]
\NormalTok{sem_wait(sem_t *s) \{}
  \NormalTok{pthread_mutex_lock(&s->m);}
  \KeywordTok{while} \NormalTok{(s->count == }\DecValTok{0}\NormalTok{) \{}
      \NormalTok{pthread_cond_wait(&s->cv, &s->m); }\CommentTok{/*unlock mutex, wait, relock mutex*/}
  \NormalTok{\}}
  \NormalTok{s->count--;}
  \NormalTok{pthread_mutex_unlock(&s->m);}
\NormalTok{\}}
\end{Highlighting}
\end{Shaded}

Wait \texttt{sem\_post} keeps calling \texttt{pthread\_cond\_signal}
won't that break sem\_wait?\\Answer: No! We can't get past the loop
until the count is non-zero. In practice this means \texttt{sem\_post}
would unnecessary call \texttt{pthread\_cond\_signal} even if there are
no waiting threads. A more efficient implementation would only call
\texttt{pthread\_cond\_signal} when necessary i.e.

\begin{Shaded}
\begin{Highlighting}[]
  \CommentTok{/* Did we increment from zero to one- time to signal a thread sleeping inside sem_post */}
  \KeywordTok{if} \NormalTok{(s->count == }\DecValTok{1}\NormalTok{) }\CommentTok{/* Wake up one waiting thread!*/}
     \NormalTok{pthread_cond_signal(&s->cv);}
\end{Highlighting}
\end{Shaded}

\subsection{Other semaphore
considerations}\label{other-semaphore-considerations}

\begin{itemize}
\itemsep1pt\parskip0pt\parsep0pt
\item
  Real semaphores implementation include a queue and scheduling concerns
  to ensure fairness and priority e.g.~wake up the highest-priority
  longest sleeping thread.
\item
  Also, an advanced use of \texttt{sem\_init} allows semaphores to be
  shared across processes. Our implementation only works for threads
  inside the same process.
\end{itemize}
