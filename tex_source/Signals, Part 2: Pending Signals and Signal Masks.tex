\chapter{Signals} is Part 1?}\label{where-is-part-1}

There is no official ``Part 1'' page. However we introduced a simple
signal() callback at the beginning of the course (e.g. {[}{[}Forking,
Part 2: Fork, Exec, Wait Kill{]}{]} )

\subsection{How can I learn more about
signals?}\label{how-can-i-learn-more-about-signals}

The linux man pages discusses signal system calls in section 2. There is
also a longer article in section 7 (though not in OSX/BSD):

\begin{verbatim}
man -s7 signal
\end{verbatim}

\subsection{What is a process's signal
disposition?}\label{what-is-a-processs-signal-disposition}

For each process, each signal has a disposition which means what action
will occur when a signal is delivered to the process. For example, the
default disposition SIGINT is to terminate it. However this disposition
can be changed by calling \texttt{signal()} (or as we will learn later)
\texttt{sigaction()} to install a signal handler for a particular
signal. You can imagine the processes' disposition to all possible
signals as a table of function pointers entries (one for each possible
signal).

The default disposition for signals can be to ignore the signal, stop
the process, continue a stopped process, terminate the process, or
terminate the process and also dump a `core' file. Note a core file is a
representation of the processes' memory state that can be inspected
using a debugger.

\subsection{Can multiple signals be
queued?}\label{can-multiple-signals-be-queued}

No - however it is possible to have signals that are in a pending state.
If a signal is pending it means it has not yet been delivered to the
process. The most common reason for a signal to be pending is that the
process (or thread) has currently blocked that particular signal.

If a particular signal, e.g.~SIGINT, is pending then it is not possible
to queue up the same signal again.

It \emph{is} possible to have more than one signal of a different type
in a pending state. For example SIGINT and SIGTERM signals may be
pending (i.e.~not yet delivered to the target process)

\subsection{How do I block signals?}\label{how-do-i-block-signals}

Signals can be blocked (meaning they will stay in the pending state) by
setting the process signal mask or, when you are writing a
multi-threaded program, the thread signal mask.

\subsection{What happens when creating a new
thread?}\label{what-happens-when-creating-a-new-thread}

The new thread inherits a copy of the calling thread's mask

\begin{Shaded}
\begin{Highlighting}[]
\NormalTok{pthread_sigmask( ... ); }\CommentTok{// set my mask to block delivery of some signals}
\NormalTok{pthread_create( ... ); }\CommentTok{// new thread will start with a copy of the same mask}
\end{Highlighting}
\end{Shaded}

\subsection{What happens when forking?}\label{what-happens-when-forking}

The child process inherits a copy of the parent's signal dispositions.
In other words, if you have installed a SIGINT handler before forking,
then the child process will also call the handler if a SIGINT is
delivered to the child.

Note pending signals for the child are \emph{not} inherited during
forking.

\subsection{What is signal disposition
?}\label{what-is-signal-disposition}

The signal disposition of a process is a table of actions. It defines
what will happen when a particular signal is delivered to a process. For
example, the default disposition of SIG-INT is to terminate the process.
The signal disposition is per process not per thread. The signal
disposition can be changed by calling signal() (which is simple but not
portable as there are subtle variations in its implementation on
different POSIX architectures and also not recommended for
multi-threaded programs) or \texttt{sigaction} (discussed later)

\subsection{What happens during exec ?}\label{what-happens-during-exec}

Remember that \texttt{exec} replaces the current image with a new
program image. In addition the signal disposition is reset. Any pending
signals are cleared.

\subsection{What happens during fork ?}\label{what-happens-during-fork}

The child process inherits a copy of the parent process's signal
disposition and a copy of the parent's signal mask.

For example if \texttt{SIGINT} is blocked in the parent it will be
blocked in the child too.\\For example if the parent installed a handler
(call-back function) for SIG-INT then the child will also perform the
same behavior.

Pending signals however are not inherited by the child.

\subsection{How do I block signals in a single-threaded
program?}\label{how-do-i-block-signals-in-a-single-threaded-program}

Use \texttt{sigprocmask}! With sigprocmask you can set the new mask, add
new signals to be blocked to the process mask, and unblock currently
blocked signals. You can also determine the existing mask (and use it
for later) by passing in a non-null value for oldset.

\begin{verbatim}
int sigprocmask(int how, const sigset_t *set, sigset_t *oldset);`
\end{verbatim}

From the Linux man page of sigprocmask,

\begin{verbatim}
SIG_BLOCK: The set of blocked signals is the union of the current set and the set argument.
SIG_UNBLOCK: The signals in set are removed from the current set of blocked signals. It is permissible to attempt to unblock a signal which is not blocked.
SIG_SETMASK: The set of blocked signals is set to the argument set.
\end{verbatim}

The sigset type behaves as a bitmap, except functions are used rather
than explicitly setting and unsetting bits using \& and \textbar{}.

It is a common error to forget to initialize the signal set before
modifying one bit. For example,

\begin{Shaded}
\begin{Highlighting}[]
\NormalTok{sigset_t set, oldset;}
\NormalTok{sigaddset(&set, SIGINT); }\CommentTok{// Ooops!}
\NormalTok{sigprocmask(SIG_SETMASK, &set, &oldset)}
\end{Highlighting}
\end{Shaded}

Correct code initializes the set to be all on or all off. For example,

\begin{Shaded}
\begin{Highlighting}[]
\NormalTok{sigfillset(&set); }\CommentTok{// all signals}
\NormalTok{sigprocmask(SIG_SETMASK, &set, NULL); }\CommentTok{// Block all the signals!}
\CommentTok{// (Actually SIGKILL or SIGSTOP cannot be blocked...)}

\NormalTok{sigemptyset(&set); }\CommentTok{// no signals }
\NormalTok{sigprocmask(SIG_SETMASK, &set, NULL); }\CommentTok{// set the mask to be empty again}
\end{Highlighting}
\end{Shaded}

\subsection{How do I block signals in a multi-threaded
program?}\label{how-do-i-block-signals-in-a-multi-threaded-program}

Blocking signals is similar in multi-threaded programs to
single-threaded programs:

\begin{itemize}
\itemsep1pt\parskip0pt\parsep0pt
\item
  Use pthread\_sigmask instead of sigprocmask
\item
  Block a signal in all threads to prevent its asynchronous delivery
\end{itemize}

The easiest method to ensure a signal is blocked in all threads is to
set the signal mask in the main thread before new threads are created

\begin{Shaded}
\begin{Highlighting}[]
\NormalTok{sigemptyset(&set);}
\NormalTok{sigaddset(&set, SIGQUIT);}
\NormalTok{sigaddset(&set, SIGINT);}
\NormalTok{pthread_sigmask(SIG_BLOCK, &set, NULL);}

\CommentTok{// this thread and the new thread will block SIGQUIT and SIGINT}
\NormalTok{pthread_create(&thread_id, NULL, myfunc, funcparam);}
\end{Highlighting}
\end{Shaded}

Just as we saw with sigprocmask, pthread\_sigmask includes a `how'
parameter that defines how the signal set is to be used:

\begin{Shaded}
\begin{Highlighting}[]
\NormalTok{pthread_sigmask(SIG_SETMASK, &set, NULL) - replace the thread's mask with given signal set}
\NormalTok{pthread_sigmask(SIG_BLOCK, &set, NULL) - add the signal set to the thread's mask}
\NormalTok{pthread_sigmask(SIG_UNBLOCK, &set, NULL) - remove the signal set from the thread's mask}
\end{Highlighting}
\end{Shaded}

\subsection{How are pending signals delivered in a multi-threaded
program?}\label{how-are-pending-signals-delivered-in-a-multi-threaded-program}

A signal is delivered to any signal thread that is not blocking that
signal.

If the two or more threads can receive the signal then which thread will
be interrupted is arbitrary!
