\subsection{What is a counting
semaphore?}\label{what-is-a-counting-semaphore}

A counting semaphore contains a value and supports two operations
``wait'' and ``post''. Post increments the semaphore and immediately
returns. ``wait'' will wait if the count is zero. If the count is
non-zero the semaphore decrements the count and immediately returns.

An analogy is a count of the cookies in a cookie jar (or gold coins in
the treasure chest). Before taking a cookie, call `wait'. If there are
no cookies left then \texttt{wait} will not return: It will
\texttt{wait} until another thread increments the semaphore by calling
post.

In short, \texttt{post} increments and immediately returns whereas
\texttt{wait} will wait if the count is zero. Before returning it will
decrement count.

\subsection{How do I create a
semaphore?}\label{how-do-i-create-a-semaphore}

This page introduces unnamed semaphores. Unfortunately Mac OS X does not
support these yet.

First decide if the initial value should be zero or some other value
(e.g.~the number of remaining spaces in an array).\\Unlike pthread mutex
there are not shortcuts to creating a semaphore - use `sem\_init'

\begin{Shaded}
\begin{Highlighting}[]
\NormalTok{sem_t s;}
\DataTypeTok{int} \NormalTok{main() \{}
  \NormalTok{sem_init(&s, }\DecValTok{0}\NormalTok{, }\DecValTok{10}\NormalTok{); }\CommentTok{// returns -1 (=FAILED) on OS X}
  \NormalTok{sem_wait(&s); }\CommentTok{// Could do this 10 times without blocking}
  \NormalTok{sem_post(&s); }\CommentTok{// Announce that we've finished (and one more resource item is available; increment count)}
  \NormalTok{sem_destroy(&s); }\CommentTok{// release resources of the semaphore}
\end{Highlighting}
\end{Shaded}

\subsection{Can I call wait and post from different
threads?}\label{can-i-call-wait-and-post-from-different-threads}

Yes! Unlike a mutex, the increment and decrement can be from different
threads.

\subsection{Can I use a semaphore instead of a
mutex?}\label{can-i-use-a-semaphore-instead-of-a-mutex}

Yes - though the overhead of a semaphore is greater. To use a semaphore:

\begin{itemize}
\itemsep1pt\parskip0pt\parsep0pt
\item
  Initialize the semaphore with a count of one.
\item
  Replace \texttt{...lock} with \texttt{sem\_wait}
\item
  Replace \texttt{...unlock} with \texttt{sem\_post}
\end{itemize}

\subsection{Can I use sem\_post inside a signal
handler?}\label{can-i-use-semux5fpost-inside-a-signal-handler}

Yes! \texttt{sem\_post} is one of a handful of functions that can be
correctly used inside a signal handler.\\This means we can release a
waiting thread which can now make all of the calls that we were
not\\allowed to call inside the signal handler itself (e.g.
\texttt{printf}).

\begin{Shaded}
\begin{Highlighting}[]
\OtherTok{#include <stdio.h>}
\OtherTok{#include <pthread.h>}
\OtherTok{#include <signal.h>}
\OtherTok{#include <semaphore.h>}
\OtherTok{#include <unistd.h>}

\NormalTok{sem_t s;}

\DataTypeTok{void} \NormalTok{handler(}\DataTypeTok{int} \NormalTok{signal)}
\NormalTok{\{}
    \NormalTok{sem_post(&s); }\CommentTok{/* Release the Kraken! */}
\NormalTok{\}}

\DataTypeTok{void} \NormalTok{*singsong(}\DataTypeTok{void} \NormalTok{*param)}
\NormalTok{\{}
    \NormalTok{sem_wait(&s);}
    \NormalTok{printf(}\StringTok{"I had to wait until your signal released me!}\CharTok{\textbackslash{}n}\StringTok{"}\NormalTok{);}
\NormalTok{\}}

\DataTypeTok{int} \NormalTok{main()}
\NormalTok{\{}
    \DataTypeTok{int} \NormalTok{ok = sem_init(&s, }\DecValTok{0}\NormalTok{, }\DecValTok{0} \CommentTok{/* Initial value of zero*/}\NormalTok{); }
    \KeywordTok{if} \NormalTok{(ok == -}\DecValTok{1}\NormalTok{) \{}
       \NormalTok{perror(}\StringTok{"Could not create unnamed semaphore"}\NormalTok{);}
       \KeywordTok{return} \DecValTok{1}\NormalTok{;}
    \NormalTok{\}}
    \NormalTok{signal(SIGINT, handler); }\CommentTok{// Too simple! See note below}

    \NormalTok{pthread_t tid;}
    \NormalTok{pthread_create(&tid, NULL, singsong, NULL);}
    \NormalTok{pthread_exit(NULL); }\CommentTok{/* Process will exit when there are no more threads */}
\NormalTok{\}}
\end{Highlighting}
\end{Shaded}

Note robust programs do not use \texttt{signal()} in a multi-threaded
program (``The effects of signal() in a multithreaded process are
unspecified.'' - the signal man page); a more correct program will need
to use \texttt{sigaction}.

\subsection{How do I find out more?}\label{how-do-i-find-out-more}

Play using a real linux system! (9/19/14: Linux-In-the-Browser project
is missing semaphore.h - this will be fixed in the next update). Read
the man pages:

\begin{itemize}
\itemsep1pt\parskip0pt\parsep0pt
\item
  {[}{[}sem\_init\textbar{}\url{http://man7.org/linux/man-pages/man3/sem_init.3.html}{]}{]}
\item
  {[}{[}sem\_wait\textbar{}\url{http://man7.org/linux/man-pages/man3/sem_wait.3.html}{]}{]}
\item
  {[}{[}sem\_post\textbar{}\url{http://man7.org/linux/man-pages/man3/sem_post.3.html}{]}{]}
\item
  {[}{[}sem\_destroy\textbar{}\url{http://man7.org/linux/man-pages/man3/sem_destroy.3.html}{]}{]}
\end{itemize}
