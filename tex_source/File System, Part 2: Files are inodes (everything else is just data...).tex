Big idea: Forget names of files: The `inode' is the file.

It is common to think of the file name as the `actual' file. It's not!
Instead consider the inode as the file. The inode holds the
meta-information (last accessed, ownership, size) and points to the disk
blocks used to hold the file contents.

\subsection{So\ldots{} How do we implement a
directory?}\label{so-how-do-we-implement-a-directory}

A directory is just a mapping of names to inode numbers.\\POSIX provides
a small set of functions to read the filename and inode number for each
entry (see below)

\subsection{How can I find the inode number of a
file?}\label{how-can-i-find-the-inode-number-of-a-file}

From a shell, use \texttt{ls} with the \texttt{-i} option

\begin{verbatim}
$ ls -i
12983989 dirlist.c      12984068 sandwhich.c
\end{verbatim}

From C, call one of the stat functions (introduced below).

\subsection{How do I find out meta-information about a file (or
directory)?}\label{how-do-i-find-out-meta-information-about-a-file-or-directory}

Use the stat calls. For example, to find out when my `notes.txt' file
was last accessed -

\begin{Shaded}
\begin{Highlighting}[]
   \KeywordTok{struct} \NormalTok{stat s;}
   \NormalTok{stat(}\StringTok{"notes.txt"}\NormalTok{, & s);}
   \NormalTok{printf(}\StringTok{"Last accessed %s"}\NormalTok{, ctime(s.st_atime));}
\end{Highlighting}
\end{Shaded}

There are actually three versions of \texttt{stat};

\begin{Shaded}
\begin{Highlighting}[]
       \DataTypeTok{int} \NormalTok{stat(}\DataTypeTok{const} \DataTypeTok{char} \NormalTok{*path, }\KeywordTok{struct} \NormalTok{stat *buf);}
       \DataTypeTok{int} \NormalTok{fstat(}\DataTypeTok{int} \NormalTok{fd, }\KeywordTok{struct} \NormalTok{stat *buf);}
       \DataTypeTok{int} \NormalTok{lstat(}\DataTypeTok{const} \DataTypeTok{char} \NormalTok{*path, }\KeywordTok{struct} \NormalTok{stat *buf);}
\end{Highlighting}
\end{Shaded}

For example you can use \texttt{fstat} to find out the meta-information
about a file if you already have an file descriptor associated with that
file

\begin{Shaded}
\begin{Highlighting}[]
   \NormalTok{FILE *file = fopen(}\StringTok{"notes.txt"}\NormalTok{, }\StringTok{"r"}\NormalTok{);}
   \DataTypeTok{int} \NormalTok{fd = fileno(file); }\CommentTok{/* Just for fun - extract the file descriptor from a C FILE struct */}
   \KeywordTok{struct} \NormalTok{stat s;}
   \NormalTok{fstat(fd, & s);}
   \NormalTok{printf(}\StringTok{"Last accessed %s"}\NormalTok{, ctime(s.st_atime));}
\end{Highlighting}
\end{Shaded}

The third call `lstat' we will discuss when we introduce symbolic links.

In addition to access,creation, and modified times, the stat structure
includes the inode number, length of the file and owner information.

\begin{Shaded}
\begin{Highlighting}[]
\KeywordTok{struct} \NormalTok{stat \{}
               \NormalTok{dev_t     st_dev;     }\CommentTok{/* ID of device containing file */}
               \NormalTok{ino_t     st_ino;     }\CommentTok{/* inode number */}
               \NormalTok{mode_t    st_mode;    }\CommentTok{/* protection */}
               \NormalTok{nlink_t   st_nlink;   }\CommentTok{/* number of hard links */}
               \NormalTok{uid_t     st_uid;     }\CommentTok{/* user ID of owner */}
               \NormalTok{gid_t     st_gid;     }\CommentTok{/* group ID of owner */}
               \NormalTok{dev_t     st_rdev;    }\CommentTok{/* device ID (if special file) */}
               \NormalTok{off_t     st_size;    }\CommentTok{/* total size, in bytes */}
               \NormalTok{blksize_t st_blksize; }\CommentTok{/* blocksize for file system I/O */}
               \NormalTok{blkcnt_t  st_blocks;  }\CommentTok{/* number of 512B blocks allocated */}
               \NormalTok{time_t    st_atime;   }\CommentTok{/* time of last access */}
               \NormalTok{time_t    st_mtime;   }\CommentTok{/* time of last modification */}
               \NormalTok{time_t    st_ctime;   }\CommentTok{/* time of last status change */}
           \NormalTok{\};}
\end{Highlighting}
\end{Shaded}

\subsection{How do I list the contents of a directory
?}\label{how-do-i-list-the-contents-of-a-directory}

Let's write our own version of `ls' to list the contents of a directory.

\begin{Shaded}
\begin{Highlighting}[]
\OtherTok{#include <stdio.h>}
\OtherTok{#include <dirent.h>}
\OtherTok{#include <stdlib.h>}
\DataTypeTok{int} \NormalTok{main(}\DataTypeTok{int} \NormalTok{argc, }\DataTypeTok{char} \NormalTok{**argv) \{}
    \KeywordTok{if}\NormalTok{(argc == }\DecValTok{1}\NormalTok{) \{}
        \NormalTok{printf(}\StringTok{"Usage: %s [directory]}\CharTok{\textbackslash{}n}\StringTok{"}\NormalTok{, *argv);}
        \NormalTok{exit(}\DecValTok{0}\NormalTok{);}
    \NormalTok{\}}
    \KeywordTok{struct} \NormalTok{dirent *dp;}
    \NormalTok{DIR *dirp = opendir(argv[}\DecValTok{1}\NormalTok{]);}
    \KeywordTok{while} \NormalTok{((dp = readdir(dirp)) != NULL) \{}
        \NormalTok{puts(dp->d_name);}
    \NormalTok{\}}

    \NormalTok{closedir(dirp);}
    \KeywordTok{return} \DecValTok{0}\NormalTok{;}
\NormalTok{\}}
\end{Highlighting}
\end{Shaded}

\subsection{How do I read the contents of a
directory?}\label{how-do-i-read-the-contents-of-a-directory}

Ans: Use opendir readdir closedir\\For example, here's a very simple
implementation of `ls' to list the contents of a directory.

\begin{Shaded}
\begin{Highlighting}[]
\OtherTok{#include <stdio.h>}
\OtherTok{#include <dirent.h>}
\OtherTok{#include <stdlib.h>}
\DataTypeTok{int} \NormalTok{main(}\DataTypeTok{int} \NormalTok{argc, }\DataTypeTok{char} \NormalTok{**argv) \{}
    \KeywordTok{if}\NormalTok{(argc ==}\DecValTok{1}\NormalTok{) \{}
        \NormalTok{printf(}\StringTok{"Usage: %s [directory]}\CharTok{\textbackslash{}n}\StringTok{"}\NormalTok{, *argv);}
        \NormalTok{exit(}\DecValTok{0}\NormalTok{);}
    \NormalTok{\}}
    \KeywordTok{struct} \NormalTok{dirent *dp;}
    \NormalTok{DIR *dirp = opendir(argv[}\DecValTok{1}\NormalTok{]);}
    \KeywordTok{while} \NormalTok{((dp = readdir(dirp)) != NULL) \{}
        \NormalTok{printf(}\StringTok{"%s %lu}\CharTok{\textbackslash{}n}\StringTok{"}\NormalTok{, dp-> d_name, (}\DataTypeTok{unsigned} \DataTypeTok{long}\NormalTok{)dp-> d_ino );}
    \NormalTok{\}}

    \NormalTok{closedir(dirp);}
    \KeywordTok{return} \DecValTok{0}\NormalTok{;}
\NormalTok{\}}
\end{Highlighting}
\end{Shaded}

\subsection{How do I check to see if a file is in the current
directory?}\label{how-do-i-check-to-see-if-a-file-is-in-the-current-directory}

For example, to see if a particular directory includes a file (or
filename) `name', we might write the following code. (Hint: Can you spot
the bug?)

\begin{Shaded}
\begin{Highlighting}[]
\DataTypeTok{int} \NormalTok{exists(}\DataTypeTok{char} \NormalTok{*directory, }\DataTypeTok{char} \NormalTok{*name)  \{}
    \KeywordTok{struct} \NormalTok{dirent *dp;}
    \NormalTok{DIR *dirp = opendir(directory);}
    \KeywordTok{while} \NormalTok{((dp = readdir(dirp)) != NULL) \{}
        \NormalTok{puts(dp->d_name);}
        \KeywordTok{if} \NormalTok{(!strcmp(dp->d_name, name)) \{}
        \KeywordTok{return} \DecValTok{1}\NormalTok{; }\CommentTok{/* Found */}
        \NormalTok{\}}
    \NormalTok{\}}
    \NormalTok{closedir(dirp);}
    \KeywordTok{return} \DecValTok{0}\NormalTok{; }\CommentTok{/* Not Found */}
\NormalTok{\}}
\end{Highlighting}
\end{Shaded}

The above code has a subtle bug: It leaks resources! If a matching
filename is found then `closedir' is never called as part of the early
return. Any file descriptors opened, and any memory allocated, by
opendir are never released. This means eventually the process will run
out of resources and an \texttt{open} or \texttt{opendir} call will
fail.

The fix is to ensure we free up resources in every possible code-path.
In the above code this means calling \texttt{closedir} before
\texttt{return\ 1}. Forgetting to release resources is a common C
programming bug because there is no support in the C lanaguage to ensure
resources are always released with all codepaths.

\subsection{An aside - A System programming pattern to clean up
resources - goto considered
useful!}\label{an-aside---a-system-programming-pattern-to-clean-up-resources---goto-considered-useful}

Note If C supported exception handling this discussion would be
unnecessary.\\Imagine your function required several temporary resources
that need to be freed before returning.\\How can we write readable code
that correctly frees resources under all code paths? Some system
programs use \texttt{goto} to jump forward into the clean up code, using
the following pattern:

\begin{Shaded}
\begin{Highlighting}[]
\DataTypeTok{int} \NormalTok{f() \{}
   \NormalTok{Acquire resource r1}
   \KeywordTok{if}\NormalTok{(...) }\KeywordTok{goto} \NormalTok{clean_up_r1}
   \NormalTok{Acquire resource r2}
   \KeywordTok{if}\NormalTok{(...) }\KeywordTok{goto} \NormalTok{clean_up_r2}

   \NormalTok{perform work}
\NormalTok{clean_up_r2:}
   \NormalTok{clean up r2}
\NormalTok{clean_up_r1:}
   \NormalTok{clean up r1}
   \KeywordTok{return} \NormalTok{result}
\NormalTok{\}}
\end{Highlighting}
\end{Shaded}

Whether this is a good thing or not has led to long rigorous debates
that have generally helped system programmers stay warm during the cold
winter months. Are there alternatives? Yes! For example using
conditional logic, breaking out of do-while loops and writing secondary
functions that perform the innermost work. However all choices are
problematic and cumbersome as we are attempting to shoe-horn in
exception handling in a language that has no inbuilt support for it.

\subsection{What are the gotcha's of using readdir? For example to
recursively search
directories?}\label{what-are-the-gotchas-of-using-readdir-for-example-to-recursively-search-directories}

There are two main gotchas and one consideration:\\The \texttt{readdir}
function returns ``.'' (current directory) and ``..'' (parent
directory). If you are looking for sub-directories, you need to
explicitly exclude these directories.

For many applications it's reasonable to check the current directory
first before recursively searching sub-directories. This can be achieved
by storing the results in a linked list, or resetting the directory
struct to restart from the beginning.

One final note of caution: \texttt{readdir} is not thread-safe! For
multi-threaded searches use \texttt{readdir\_r} which requires the
caller to pass in the address of an existing dirent struct.

See the man page of readdir for more details.

\subsection{How I do determine if a directory entry is a
directory?}\label{how-i-do-determine-if-a-directory-entry-is-a-directory}

Ans: Use \texttt{S\_ISDIR} to check the mode bits stored in the stat
structure

And to check if a file is regular file use \texttt{S\_ISREG},

\begin{Shaded}
\begin{Highlighting}[]
   \KeywordTok{struct} \NormalTok{stat s;}
   \KeywordTok{if} \NormalTok{(}\DecValTok{0} \NormalTok{== stat(name, &s)) \{}
      \NormalTok{printf(}\StringTok{"%s "}\NormalTok{, name);}
      \KeywordTok{if} \NormalTok{(S_ISDIR( s.st_mode)) puts(}\StringTok{"is a directory"}\NormalTok{);}
      \KeywordTok{if} \NormalTok{(S_ISREG( s.st_mode)) puts(}\StringTok{"is a regular file"}\NormalTok{);}
   \NormalTok{\} }\KeywordTok{else} \NormalTok{\{}
      \NormalTok{perror(}\StringTok{"stat failed - are you sure I can read this file's meta data?"}\NormalTok{);}
   \NormalTok{\}}
\end{Highlighting}
\end{Shaded}

